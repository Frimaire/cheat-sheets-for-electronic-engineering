\documentclass[b5paper,10pt,UTF8]{ctexart}

\usepackage[left=1.27cm,right=1.27cm,top=1.52cm,bottom=1.27cm]{geometry}
\usepackage{float}
\usepackage{amsmath}
\usepackage{array}
\usepackage{fancyhdr}
\usepackage{caption}
\usepackage{multirow}

\pagestyle{fancy}

\lhead{}
\chead{单边z变换}
\rhead{}
\lfoot{\copyright Frimaire 2019}
\cfoot{}
\rfoot{CC BY}

\renewcommand\arraystretch{1}
\newcommand{\dqcoll}[1]{\multicolumn{2}{|>{\(\displaystyle{}}c<{\)}||}{#1}}
\newcommand{\dqcolr}[1]{\multicolumn{2}{>{\(\displaystyle{}}c<{\)}|}{#1}}
\newcommand{\tmath}[1]{\(\displaystyle{} #1\)}
\setlength{\textfloatsep}{1pt}

\begin{document}

\begin{table}
  \centering
  \begin{tabular}{|>{\centering\(\displaystyle{}}p{0.20\textwidth}<{\)}|>{\centering\(\displaystyle{}}p{0.23\textwidth}<{\)}||>{\centering\(\displaystyle{}}p{0.18\textwidth}<{\)}|>{\centering\(\displaystyle{}}p{0.30\textwidth}<{\)}|}
    %Command "centering" will redefine the doublebackslash, which makes command "arraybackslash" essential.
    \hline
    f[n] & F(z) & f[n] & F(z) \arraybackslash\\
    \hline
    \frac{1}{2\pi{}j}\oint{}F(z)z^{n - 1}dz & \sum_{n = -\infty}^{+\infty}f[n]z^{-n} & \delta{}[n] & 1 \arraybackslash\\
    \hline
    af_1[n] + bf_2[n] & aF_1(z) + bF_2(z) & u[n] & \frac{1}{1 - z^{-1}} \arraybackslash\\
    \hline
    f[n - 1] & z ^ {-1} F(z) + f[-1] & a^nu[n] & \frac{1}{1 - az^{-1}} \arraybackslash\\
    \hline
    f[n + 1] & z F(z) - z f[0] & na^nu[n] & \frac{az^{-1}}{(1 - az^{-1})^2} \arraybackslash\\
    \hline
    z_0^n f[n] & F(\frac{z}{z_0}) & n^2a^nu[n] & \frac{az^{-1}(1 + az^{-1})}{(1 - az^{-1})^3} \arraybackslash\\
    \hline
    \begin{cases}
      f[m] & n = km \\
      0 & n \ne km
    \end{cases} & F(z^k) & C_{n + m - 1}^{m - 1}a^{n}u[n] & \frac{1}{(1 - az^{-1})^m} \arraybackslash\\
    \hline
    f[n] - f[n - 1] & (1 - z^{-1})F(z) - f[-1] & \cos{}(\omega_0n)u[k] & \frac{1 - \cos(\omega_0)z^{-1}}{1 - 2\cos(\omega_0)z^{-1} + z^{-2}} \arraybackslash\\
    \hline
    nf[n] & -z\frac{\mathrm{d}}{\mathrm{d}z}F(z) & \sin{}(\omega_0n)u[k] & \frac{\sin(\omega_0)z^{-1}}{1 - 2\cos(\omega_0)z^{-1} + z^{-2}} \arraybackslash\\
    \hline
    f_1[n] * f_2[n] & F_1(z)F_2(z) & a^n\cos{}(\omega_0n)u[k] & \frac{1 - a\cos(\omega_0)z^{-1}}{1 - 2a\cos(\omega_0)z^{-1} + a^2z^{-2}} \arraybackslash\\
    \hline
    \sum_{k = 0}^{n}x[k] & \frac{1}{1 - z^{-1}}F(z) & a^n\sin{}(\omega_0n)u[k] & \frac{a\sin(\omega_0)z^{-1}}{1 - 2a\cos(\omega_0)z^{-1} + a^2z^{-2}} \arraybackslash\\
    \hline
    \dqcoll{f[0] = \lim_{z\to\infty}F(z)} & \dqcolr{f[+\infty] = \lim_{z\to{}1}(1 - z^{-1})F(z)} \arraybackslash\\
    \hline
  \end{tabular}
\end{table}

\paragraph{真分式的部分分式展开}

\begin{displaymath}
F(w) = \frac{b_{n - 1}s^{n - 1}+\cdots+b_1s+b_0}{(1 - \rho_1w)^{\sigma_1}(1 - \rho_2w)^{\sigma_2}\cdots(1 - \rho_rw)^{\sigma_r}} = \sum_{n = 1}^{r}\sum_{m = 1}^{\sigma_n}\frac{B_{nm}}{(1 - \rho_nw)^m}
\end{displaymath}

\begin{displaymath}
  B_{nm} = \frac{1}{(\sigma_n - m)!}(-\rho_n)^{-(\sigma_n - m)}\cdot\frac{\mathrm{d}^{\sigma_n - m}}{\mathrm{d}w^{\sigma_n - m}}((1 - \rho_nw)^{\sigma_n}F(w)|_{w = \rho_n^{-1}}
\end{displaymath}

\paragraph{朱利判据}

\begin{displaymath}
  D(w) = a_nw^n + a_{n - 1}w^{n - 1} + \cdots + a_2w^2 + a_1w + a_0
\end{displaymath}

\begin{table}
  \centering
    \begin{tabular}{|>{\(\displaystyle{}}c<{\)}|>{\(\displaystyle{}}c<{\)}|>{\(\displaystyle{}}c<{\)}|>{\(\displaystyle{}}c<{\)}|>{\(\displaystyle{}}c<{\)}|>{\(\displaystyle{}}c<{\)}|>{\(\displaystyle{}}c<{\)}|}
      \hline
      a_n & a_{n - 1} & a_{n - 2} & \cdots & a_2 & a_1 & a_0 \arraybackslash\\
      \hline
      a_0 & a_1 & a_2 & \cdots & a_{n - 2} & a_{n - 1} & a_{n} \arraybackslash\\
      \hline
      b_{n - 1} = \begin{vmatrix}
                  a_n & a_0 \\
                  a_0 & a_n
                \end{vmatrix} & b_{n - 2} = \begin{vmatrix}
                  a_n & a_1 \\
                  a_0 & a_{n - 1}
                \end{vmatrix} & b_{n - 3} & \cdots & b_1 & b_0 & \arraybackslash\\
      \hline
      b_0 & b_1 & b_2 & \cdots & b_{n - 2} & b_{n  - 1} & \arraybackslash\\
      \hline
      c_{n - 1} = \begin{vmatrix}
        b_{n - 1} & b_0 \\
        b_{0} & b_{n - 1}
                \end{vmatrix} & c_{n - 2} = \begin{vmatrix}
        b_{n - 1} & b_1 \\
        b_{0} & b_{n - 2}
      \end{vmatrix}
                      & c_{n - 3} & \cdots & c_0 & & \arraybackslash\\
      \hline
      c_0 & c_1 & c_2 & \cdots & c_{n - 2} & & \arraybackslash\\
      \hline
      \vdots & \vdots & \vdots & \vdots & \vdots & & \arraybackslash\\
      \hline
      q_2 & q_1 & q_0 & & & &\\
      \hline
  \end{tabular}
\end{table}

当$D(1) > 0$,$(-1)^nD(-1) > 0$,且奇数行首列元素大于该行末项元素的绝对值时,$D(w) = 0$的根全部在单位圆内。

\end{document}
